\documentclass[a4paper]{article}

\usepackage{titlesec}
\usepackage{listings}
\usepackage{graphicx}
\usepackage{color}
\usepackage[margin=1.2in]{geometry}

\usepackage{amssymb,amsmath}

\usepackage[colorlinks=true,
            allcolors=green]{hyperref}

\newcommand{\kora}{%
(\raisebox{0.5em}{\rotatebox{-45}{)}}$^{\circ}{\scriptscriptstyle\Box}^{\circ}$)\raisebox{0.5em}{\rotatebox{-45}{)}}\rotatebox{90}{)}\raisebox{0.2em}{\LARGE \_\hskip-.1em\textvisiblespace\hskip-.1em\_}
}

\title{Analysis of Simple Factory Pattern and Factory Pattern}

\makeatletter
\renewcommand{\maketitle}{\bgroup\setlength{\parindent}{0pt}
    \begin{center}

        \textbf{\huge\@title} \\[12pt]

        \par\vspace{\baselineskip}
    \end{center}

\egroup}
\makeatother

\titleformat{\section}
{\large\bfseries}
{\thesection}
{0.5em}
{}

\lstset{
    frame=single
}

\begin{document}
\maketitle

\section{Analysis of Assignment}

In this assignment, we are required to analyse the condition of the reduction
from a \textit{Factory Pattern} to a \textit{Simple Factory Pattern}. Considering
answering ``When'' and ``How'' to accomplish the degradation, one need three
basic theory to explain this work. First is the paradigm of creating \textit{Simple Factory Pattern},
second is the paradigm of implementing \textit{Factory Pattern} and the third is
the detailed way we harbour to make the former evolve to latter. Note that all
discussions are based on condition discribed in section \ref{define}.

\section{Simple Factory Pattern and Factory Pattern}\label{define}

To illustrate the problem discussed above, we visualize the relationship
of classes, concrete objects and interfaces of the two patterns as below (Figure \ref{simple_fty}
for \textit{Simple Factory Pattern } and figure \ref{fty} for \textit{Factory Pattern} with description in
section \ref{code}). To be specific, the former concretes its case switching in
the unmoving class in the \texttt{SimpleFactory} while we extract the ``Factory'' level of
the second design to substantiate an interface (as well as an abstract class in some degree),
with the first design, it's more convenient for the client but is a trade-off with scalability.

\begin{figure}[h]
\begin{center}
\includegraphics[width=0.6\textwidth]{simple_fty_pattern.png}
\end{center}
\caption{Simple Factory Pattern}
\label{simple_fty}
\end{figure}

As it shows, \textit{Factory Pattern} is different from the aspect of ``Factory'' level,
During a single runtime, one can only produce certain type of products,
and to add new product (no matter the brand or process of producing is different), programmer should
modify a concrete class like \texttt{SimpleFactory}.

\begin{figure}[h]
\begin{center}
\includegraphics[width=0.6\textwidth]{fty_pattern.png}
\end{center}
\caption{Factory Pattern}
\label{fty}
\end{figure}

Paradigm is changed when it comes to \textit{Factory Pattern}. There exists abstraction of products (\texttt{Product}),
as well as the delayed implementing of ``case switching'' in all kinds of creator as an interface (\texttt{Factory}).
With this pattern, one can build the same amount of sorts of products with limited types of classes compared
to the simple pattern, but altering the factorys (various channels of producing things) to create
multiple products at runtime
\footnote{It's also supported for \textit{Simple Factory Pattern} to build various objects at runtime  with a wrapper with a factory setter.} with ease.

As for editing, to create a new class (new factory), one only have to
make the new class inherited from the abstraction (\texttt{Dependency Reverse}) and
should and will not reconstruct the backbone of the original design (\texttt{Open Close}).

\section{Condition of Reduction} \label{condition}

Now it's easier to discuss the situation we are faced with, to begin with,
if we are based on ``producing products'', and if there are:

\begin{itemize}
    \item Certain and limited types of products that the factory should
        produce with no extraction of the abstraction.
        ``Factory'' downgrades to ``Simple Factory'', automatically.
    \item The number of abstraction level is only ONE, like all the products are ``TV'', among which differences are only various brands,
        then only implementing the \texttt{TVFactory} should be fine
        \footnote{\texttt{AppleTV} and \texttt{HaierFridge} have an abstraction level of brand,
        as well as the types of machine, totally two levels.}.
\end{itemize}

On the basis of ``code modifying'', from the perspective of a programmer, to answer question ``When'', if situations are:

\begin{itemize}
    \item Only limited amonut of classes, with a pure design of structure,
        removing the abstraction causes no harm to structure and lead to limited time and labor.
\end{itemize}

\section{Means of Alteration}\label{means}

As illustrated above, one can reduce the \textit{Factory Pattern} to \textit{Simple Factory Pattern}
as long as the conditions is set up. More specifically, to make the former perform like
the latter at code-editing time or at runtime, we can, repectively:

\begin{itemize}
    \item\label{itm:programmer} Ignore the abstraction (should still do inheritance, or the code couldn't compile), then we can also create various brands and
        multiple types of machines with ease. Once a factory contains two or more levels of abstraction,
        say \texttt{AppleTVOLEDScreen} (see \ref{code-pro}, totally three levels \kora), then it works as a simple factory (ONE factory contains ALL!!!).
    \item\label{itm:client} Ignore the concrete interface. Adopt \texttt{AppleFactory} to create an \texttt{AppleTV} as in \ref{code-cli} instead of
        using \texttt{Factory} in section \ref{code}, then it performs just like a simple factory
        \footnote{\textit{Simple Factory Pattern} uses \texttt{SimpleFactory} to build a simple-product.}
        \footnote{Duck test: If it looks like a duck, swims like a duck, and quacks like a duck, then it probably is a duck.}.
\end{itemize}

\section{Conclusion}

To conclude, we mentioned the details of two patterns with concrete example in section \ref{define} that
\textit{Simple Factory Pattern} contains only an abstraction of the products,
and the \textit{Factory Pattern} constructs itself with an abstraction of
the factory itself, providing extendability and ability of alteration, avoiding
modifications of the structurized code of the pattern or the interface.

We illustrated the possible conditions in section \ref{condition} of the required reduction
and multiple channels of modifying the code from the perspective of
the client \ref{itm:client} and the programmer \ref{itm:programmer} in section \ref{means},
described in \ref{code-cli} and \ref{code-pro} repectively.
Hopefully we stated the problem and solutions correctly and accurately.

\newpage

\section{Appendix}\label{code}

\subsection{Simple Factory}

\begin{lstlisting}[language=java]
    class SimpleFactory {
        public Product createProduct(String type) {
            Product newProduct;
            /* case switching */
            return newProduct;
        }
    }

    public static void main(String[] args) {
        SimpleFactory fty = new SimpleFactory();
        Product p = fty.createProduct("type");
    }
\end{lstlisting}

\subsection{Factory}

\begin{lstlisting}[language=java]
    abstract class Factory {
        protected abstract Product createProduct(String type);
    }

    class AppleFactory extends Factory {
        public Product createProduct(String type) {
            Product newProduct;
            /* case switching */
            return newProduct;
        }
    }

    class HaierFactory extends Factory {
        public Product createProduct(String type) {
            Product newProduct;
            /* case switching */
            return newProduct;
        }
    }

    public static void main(String[] args) {
        Factory[] fty = new Factory[2];
        fty[0] = new AppleFactory();
        fty[1] = new HaierFactory();
        Product p1 = fty.createProduct("type");
        Product p2 = fty.createProduct("type");
    }
\end{lstlisting}

\newpage

\subsection{Factory Reduction 1}\label{code-cli}

\begin{lstlisting}[language=java]
    /* . . . */

    public static void main(String[] args) {
        AppleFactory appleFty = new AppleFactory();
        HaierFactory haierFty = new HaierFactory();
        Product p1 = AppleFactory.createProduct("type");
        Product p2 = HaierFactory.createProduct("type");
    }
\end{lstlisting}

\subsection{Factory Reduction 2}\label{code-pro}
\begin{lstlisting}[language=java]
    /* . . . */

    class AppleFactory extends Factory {
        public Product createProduct(String type) {
            Product newProduct;
            if (type.equals("AppleTVLEDScreen")) {
                /* . . . */
            } else if (type.equals("AppleTVOLEDScreen")) {
                /* . . . */
            } else if (type.equals("AppleFridge...")) {
                /* . . . */
            } else {
                /* . . . */
            }
            return newProduct;
        }
    }

    /* . . . */
\end{lstlisting}

\end{document}
