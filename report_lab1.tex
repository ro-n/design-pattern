\documentclass[a4paper]{article}

\usepackage{titlesec}
\usepackage{listings}
\usepackage{graphicx}
\usepackage{color}
\usepackage[margin=1.2in]{geometry}
\usepackage{algorithm}
\usepackage{algorithmic}
\usepackage{commath}
\usepackage{moreverb}
\usepackage{amssymb,amsmath}

\usepackage[colorlinks=true,
            allcolors=green]{hyperref}

\newcommand{\kora}{%
(\raisebox{0.5em}{\rotatebox{-45}{)}}$^{\circ}{\scriptscriptstyle\Box}^{\circ}$)\raisebox{0.5em}{\rotatebox{-45}{)}}\rotatebox{90}{)}\raisebox{0.2em}{\LARGE \_\hskip-.1em\textvisiblespace\hskip-.1em\_}
}

\title{Analysis of Lab\_\#1: Hold Reference and Do Dirty Work,
    A Bidirectional Observer Pattern}

\makeatletter
\renewcommand{\maketitle}{\bgroup\setlength{\parindent}{0pt}
    \begin{center}

        \textbf{\Large\@title} \\[12pt]


        \par\vspace{\baselineskip}
    \end{center}

\egroup}
\makeatother

\titleformat{\section}
{\large\bfseries}
{\thesection}
{0.5em}
{}

\lstset{
    frame=single
}

\begin{document}
\maketitle

\section{Analysis of Lab}%
\label{sec:analysis_of_lab}

In this lab, as we will be realizing \textbf{case 4},
we are required to implement a stock-investor program
which is capable of providing the abstraction of behaviour like
``customer buys a stock'', ``stock company raises up price'' and
``customer sells the stock to earn some money (in response to
the changing price)''. There are already some patterns like
\textit{Observer Pattern} that sets the mechanism for
single direction notifying and being updated. However,
the former pattern is lack of effective strategy for
situations resembling, say the back updating requirement of
the stock price changing for the company when they come up with the infomation
that their products are popular as they can force it up or vice versa, aiming at
which the former pattern can never cope with, so we propose
a bidirectional pattern, that is to say, to hold references on
both sides, solving this problem. Note that we use Java language in
this project and features of components will be in certain format
\footnote{\underline{package}, \texttt{class}, \textbf{file}, \textsl{method}, \textit{pattern}}

\section{Overall Design}%
\label{sec:overall_design}

In this section, we will illustrate the overall design of our lab work.
From implementing abstrations on basic \textit{Observer Pattern}
to modifying class contents to fix the back calling without reconstructing
the concrete frame of the subject-observer structure.

\subsection{Understanding Observer Pattern}%
\begin{figure}[h]
    \centering
    \includegraphics[width=0.5\textwidth]{obs.png}
    \caption{UML Diagram of \textit{Observer Pattern}, there are
    totally two interfaces (abstract classes) with a relationship of
    composition, one should implement interfaces at left side of
    the connection if it focuses on functions of being observered and
    maintaining a collection of its observers, while the other should
    expose an interface for being notified.
}
    \label{fig:obs_pattern}
\end{figure}
\label{sub:understanding_observer_pattern}

So basically the \textit{Observer Pattern} simply emphasize a
loose coupling connection between the two. Notice there it is only one channel
for single way message passing so long as the subject does not expose an \textsl{update()} interface
and adding new observer leads to no alteration in original code, conforming to open-close principle.

\subsection{Why Original Observer Does Not Work}%
\label{sub:why_original_observer_does_not_work}

It seems hard to implement a single directional pattern on concrete examples
like meeting the requirements in this lab which is in need of
a message response for situations like ``client selling the stock to earn
money'' (may incur a chain of calling). Thus, we propose a
\textit{Bidirectional Observer Pattern} to solve the case 4.

\subsection{Bidirectional Observer Pattern}%
\label{sub:bidirectional_observer_pattern}

Conceptually a \textit{Bidirectional Observer Pattern} is an idea
consisting of holding a series of references for passing information across classes
and exposing specific public interfaces for other subjects
at sides so long as the individual class is hoping to sending
messages and being called with notifications from others at the same time.

 Say the \texttt{Subject} in \textit{Observer Pattern}
maintaining a collection of observers  in side of its attribute list:
$\{\left(S, O_i\right)|O_i \in \{ O_1, O_2, \ldots, O_n\}\}$
\footnote{We denote $S$, $L$ and $O$ to be subject, logic and the observer respectively.}, while the sum
relationship of subjects is a collection: $C_{S \to O} = \{\left(S_j, O_i\right)|O_i \in \{ O_1, O_2, \ldots, O_n\},
S_j \in \{ S_1, S_2, \ldots, S_m\}\}$ for there are $n$ observers and $m$ subjects in
the space, for at maximum of $n \times m$ tuples.

On meeting the requirements of the lab, \texttt{Observer} also
maintaining a set of references in side of its attribute list (\textsl{HashMap} here, for
an additional mapping from \texttt{SubjectStock} to \texttt{ClientLogic}), as
$\{ \left( O_i, \left( S_j, L_k \right)  \right) | \left( S_j, O_i \right) \in C_{S \to O} \}
 \}$ ,where $ k = j $ and $L_k \in \{ L_1, L_2, \ldots, L_{mn} \} $.
To be precise, there are $2 \times m \times n$ tuples in our design, without considering
the mapping subset.

\section{Method}%
\label{sec:method}

In this section, we mainly focus on detailed implementation of our pattern. We are trying to
illustrate some key components of projects and pieces of tricky code.
Also, disadvantages of this design will be discussed.

\subsection{Key Components and Detailed Design}%
\label{sub:key_components_and_detailed_design}

\subsubsection{Core Classes}

Considering key components, we've got (\underline{default package}):

\begin{itemize}
    \item \texttt{SubjectStock} described in \ref{sub:stocksubject}, the only task for this class
        is to maintain a list of \texttt{ObserverClient} in \textsl{ArrayList} inside
        of the class and expose a set of interfaces for the client to check price and do unregistering, \textit{etc}.
    \item \texttt{ObserverClient} is described in \ref{sub:clientobserver}. The only task for this class
        is to maintain a collection of mappings from \texttt{SubjectStock} to \texttt{ClientLogic} and expose
        a set of interfaces for the stock to send messages and fetch object status.
    \item \texttt{ClientLogic}, the value part of the \textsl{HashMap}
        (see section \ref{sub:clientlogic}),
        is responsible for maintaining the selling / holding strategy
        for the stocks that the bargainers already bought in, which means to
        return a bool value to represent: whether the client want a
        notification or whether it would sell the stock in response to
        the fluctuations of prices at current state.
\end{itemize}

\subsubsection{Strategies and Constraints}

On meeting demands of this lab (to sell or hold), several strategies should
be set:

\begin{itemize}
    \item Mapping from sold stocks to current price (see
        \textbf{StockThresholdMapping.java} in \underline{helper}),
        the method provides useful functions $amount \to price$ with a
        an average step function.
    \item Mapping from the threshold value $\abs{priceOld - priceNew}$
        representing certain thresh that the customer
        want to be notified with to a bool value, is described in
        \textbf{ClientThresholdMappingUpdating.java}.
    \item Being similar to above, there is also a function
        $int \to bool$ answering whether the customer should
        sell the stock or hold.
\end{itemize}

If we want to avoid situations resembling ``customer purchases
a stock with a selling-threshold that too low to
prevent a reselling immediately once the bargain is done as
he thinks the current price is high enough'',
we have to give a constraint to the conditions on not
to sell stock if the customer harbours only one stock.

\subsubsection{Tricky Implementation}

So we've got $S \to C$ in each instance of \texttt{SubjectStock}
and $C \to \left(S, L \right) $ in \texttt{ObserverClient}, now
that a stock hopes to notify a customer (client), how could he
possibly make that? Obviously using a public interface (if in Java)
offered by the client solves this case. Further more, the client
makes the same way to send message back as \texttt{SubjectStock} just
gives \textsl{register(Observer newObs)} (see \ref{sub:stocksubject}).
When being called \textsl{update()}, the client side just makes
use of the \textsl{HashMap$<$S, L$>$},
reacting with various behaviour repectively.

The extraction of \texttt{ClientLogic} simplified the code structure with
an independent instance for each of the $C \to S$. Maintaining the client's
bargaining strategy and amounts of harbouring the, stock receives
different answers from the client. Let's make it more straight forward,
the chain of calling should be like this:


\begin{algorithm}\label{algorithm}
\caption{Chain of Recursive Calling}
\begin{algorithmic}[1]
    \REQUIRE $S \in C_{S}, C \in C_{C}, L \in C_{L}$
    \STATE initialize $HashMap<S, L>$ and $Set<C>$ with $\emptyset$
    \STATE customer $C$ buys a stock $S$
    \STATE add $S$ and $L$ to $HashMap$
    \STATE add $C$ to $Set$
    \FORALL{$S$ in $HashMap$}
        \IF{$L$ thinks $C_{i}$ should sell \AND size of $HashMap$ $>$ 1}
        \STATE remove $S_{i}$ and $L_{i}$ from $HashMap$
        \STATE remove $C_{i}$ from $Set$, jump to $11$
        \ENDIF
    \ENDFOR
    \IF{\NOT $L$ amounts of stocks have changed}
        \STATE jump to $19$
    \ENDIF
    \FORALL{$C$ in $Set$}
        \IF{$L$ thinks $C$ should be notified}
        \STATE notify $C_{i}$, jump to $5$
        \ENDIF
    \ENDFOR
    \STATE done
\end{algorithmic}
\end{algorithm}

What's more is the postponed removal of items in collections as it
avoid a concurrent exception frequently happened in situations when
a internal removal is called inside a loop of a set of itself. The
details of the trick is described in \textbf{ObserverClient.java},
method \textsl{update()}.

\subsection{Tradeoff of Bidirection}%
\label{sub:tradeoff_of_bidirection}

As for bidirectional connection in between the two classes,
there are lots of references pointing from one side to the other side.
Actually those relationships lead to a tight coupling, whose
methods seems hard to decompose or alternate with new methods
inside of which.

Although it breaks the principle encapsulation in some degree,
the concrete task wouldn't need too much of the robust features
we learned from our courses, let alone extracting a higher level abstraction
than the original structure if not necessary
\footnote{Occam's razor: Entities should not be multiplied without necessity.}.

Apart from the tight coupling of the design, there somhow exists
obstacle in the way users are trying to understand, say the
chain of recursive calling and constraints in the \textsl{update()}.

\section{Experiment}%
\label{sec:experiment}

Testing on cases which show ablation of conditions such as selling at certain thresholds,
being notified at various thresholds, individual price strategy on each stock company,
\textit{etc}, results are pasted from the console:

\begin{boxedverbatim}
At this notification A's price is from 10 to 50:
D harbours 1 stock(s) (selling thresh: 0 for A) is holding A at 50
At this notification C's price is from 0 to 25:
D harbours 2 stock(s) (selling thresh: 0 for A) is selling A at 50
D harbours 1 stock(s) (selling thresh: 45 for C) is holding C at 25
At this notification B's price is from 5 to 18:
D harbours 2 stock(s) (selling thresh: 200 for B) is holding B at 18
D harbours 2 stock(s) (selling thresh: 45 for C) is holding C at 25
At this notification C's price is from 25 to 100:
D harbours 2 stock(s) (selling thresh: 200 for B) is holding B at 18
D harbours 2 stock(s) (selling thresh: 45 for C) is selling C at 100
At this notification C's price is from 100 to 50:
E harbours 1 stock(s) (selling thresh: 90 for C) is holding C at 50
At this notification C's price is from 50 to 100:
E harbours 1 stock(s) (selling thresh: 90 for C) is holding C at 100
F harbours 1 stock(s) (selling thresh: 200 for C) is holding C at 100
At this notification B's price is from 18 to 31:
D harbours 1 stock(s) (selling thresh: 200 for B) is holding B at 31
E harbours 2 stock(s) (selling thresh: 40 for B) is holding B at 31
E harbours 2 stock(s) (selling thresh: 90 for C) is selling C at 100
\end{boxedverbatim}

\section{Conclusion}%
\label{sec:conclusion}

In this report of the lab, we illustrated basic paradigm
of the \textit{Observer Pattern} with UML diagram \ref{fig:obs_pattern} in
section \ref{sec:overall_design}. Then we analysed the capacity
of this pattern in \ref{sub:why_original_observer_does_not_work}
and proposed our method concisely in \ref{sub:bidirectional_observer_pattern}.
Meanwhile, we introduced our design with core components and
illustrated some features of our design according to the
requirements of the lab in \ref{sub:key_components_and_detailed_design}.
We also mentioned there are some drawback of our design
in \ref{sub:tradeoff_of_bidirection}. At the
end, we demonstrated the results, saying that we had met all demands.

\newpage

\section{Appendix}%
\label{sec:appendix}

\subsection{StockSubject}%
\label{sub:stocksubject}

\begin{lstlisting}[language=java]
/* doing import */

public class SubjectStock implements Subject {
        /* attribute list declaration */
	private ArrayList<Observer> arr;

	public SubjectStock (String name, int priceLow,
            int priceHigh, int[] amountThresh) {
            /* constructor */
	}

        /* public getter for client */
	public int curPriceGetter() {}
	public String getName() {}

	@Override
	public void register(Observer newObs) {
            /* append observer into list */
            /* doing price changing */
	    notifyAllObservers();
	}

	@Override
	public void unregister(Observer rmObs) {
            /* remove observer from list */
            /* doing price changing */
	    if (this.arr.size() > 0) notifyAllObservers();
	}

	@Override
	public void notifyAllObservers() {
            /* print price changing */
            /* loop through list,
            * notify observer if they want to be */
	}
}
\end{lstlisting}

\newpage

\subsection{ClientObserver}%
\label{sub:clientobserver}

\begin{lstlisting}[language=java]
/* doing import */

public class ObserverClient implements Observer {
        /* attribute list declaration */
	private HashMap<SubjectStock, ClientLogic> mapStock;

        /* constructor */
	public ObserverClient(String name) {}

        /* buy stock, core method */
	public void buyNewStock(SubjectStock stock, int amount,
            int threshSell, int threshUpdate) {
            mapStock.put(stock, ...);
            stock.register(this);
	}

        /* public getter for stock */
	public int amountGetter(SubjectStock stock) {}
	public boolean hopingToBeNotified(SubjectStock stock,
            int gap) {}

	@Override
        /* update, core method */
	public void update() {
            /* loop through HashMap<>
            * if want to sell and harbour more than 1:
            * sell stock, remove stock from HashMap
            * and itself in arr in SubjectStock
            * else:
            * holding stock */
	}
}
\end{lstlisting}

\subsection{ClientLogic}%
\label{sub:clientlogic}

\begin{lstlisting}[language=java]
class ClientLogic {
        /* attribute list */

        /* constructor */
	ClientLogic(int amount, int threshSell, int threshUpdate {}

        /* public getter for ClientObserver for conditions */
	public boolean considerSellingAt(int curPrice) {}
	public boolean considerBeingNotifiedAt(int gap) {}

        /* public getter for logic of client */
	public int getAmount() {}
	public int getThresh() {}
}
\end{lstlisting}

\end{document}
